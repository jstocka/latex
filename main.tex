% !TeX encoding = UTF-8
\documentclass[12pt,a4paper]{report}

% Ustawienia języka i kodowania
\usepackage[T1]{fontenc}
\usepackage[utf8]{inputenc}
\usepackage[polish,english]{babel}
\usepackage{csquotes}
\usepackage{titlesec}

% Ustawienia strony zgodne z wymaganiami
\usepackage[
    a4paper,
    top=2.5cm,
    bottom=2.5cm,
    left=3.5cm,
    right=2cm,
    includehead
]{geometry}

% Ustawienia akapitu i tekstu
\usepackage{setspace}
\onehalfspacing
\usepackage{indentfirst}
\usepackage{ragged2e}
\justifying

% Numeracja stron i nagłówki
\usepackage{fancyhdr}
\fancypagestyle{plain}{%
    \fancyhf{}% Clear header/footer
    \fancyfoot[R]{\thepage}% Page number in bottom right
    \renewcommand{\headrulewidth}{0pt}% No header rule
}
\pagestyle{plain}

% Bibliografia
\usepackage[
    backend=biber,
    style=numeric,
    sorting=none,
    maxbibnames=99,
    maxcitenames=2,
    giveninits=false
]{biblatex}
\addbibresource{bibliografia.bib}

% Grafika i podpisy
\usepackage{graphicx}
\usepackage{caption}
\usepackage{subcaption}
\usepackage{enumitem}

% Spisy
\usepackage[titles]{tocloft}
\renewcommand{\cftdot}{}
\renewcommand{\cftsecleader}{\cftdotfill{\cftdotsep}}

% Listingi kodu
\usepackage{listings}
\usepackage{xcolor}
\lstset{
    basicstyle=\ttfamily\small,
    numbers=left,
    numberstyle=\tiny,
    numbersep=5pt,
    breaklines=true,
    frame=single,
    backgroundcolor=\color{white},
    keywordstyle=\color{blue},
    commentstyle=\color{green!60!black},
    stringstyle=\color{red},
    showstringspaces=false,
    inputencoding=utf8,
    extendedchars=true,
    literate={ą}{{\k{a}}}1
            {ć}{{\'c}}1
            {ę}{{\k{e}}}1
            {ł}{{\l{}}}1
            {ń}{{\'n}}1
            {ó}{{\'o}}1
            {ś}{{\'s}}1
            {ź}{{\'z}}1
            {ż}{{\.z}}1
            {Ą}{{\k{A}}}1
            {Ć}{{\'C}}1
            {Ę}{{\k{E}}}1
            {Ł}{{\L{}}}1
            {Ń}{{\'N}}1
            {Ó}{{\'O}}1
            {Ś}{{\'S}}1
            {Ź}{{\'Z}}1
            {Ż}{{\.Z}}1
}

% Matematyka
\usepackage{amsmath}
\usepackage{amsthm}
\usepackage{amssymb}

% Hiperłącza
\usepackage[hidelinks]{hyperref}
\urlstyle{same}

\makeatletter
% Strona tytułowa
\renewcommand{\maketitle}{
    \begin{titlepage}
    \pagestyle{empty}
    \begin{center}
        \Large
        UNIWERSYTET KAZIMIERZA WIELKIEGO\\
        KOLEGIUM IV\\[0.5cm]
        INSTYTUT KOMUNIKACJI SPOŁECZNEJ I MEDIÓW\\[2cm]
        
        \large
        Julia Stocka\\[0.5cm]
        
        98057\\[2cm]
        
        \textbf{Mechaniki w wybranych grach rytmicznych oraz ich wpływ \\ na immersję i flow}\\[4cm]
        
        \normalsize
        \raggedleft
         Praca licencjacka napisana pod kierunkiem\\
        doktora Miłosza Markockiego\\[4cm]
        \centering
        BYDGOSZCZ 2025\\[1cm]
    \end{center}
    \end{titlepage}

    % Druga strona tytułowa po angielsku
    \begin{titlepage}
    \pagestyle{empty}
    \begin{center}
        \Large
        UNIWERSYTET KAZIMIERZA WIELKIEGO\\
        KOLEGIUM IV\\[0.5cm]
        INSTYTUT KOMUNIKACJI SPOŁECZNEJ I MEDIÓW\\[2cm]
        
        \large
        Julia Stocka\\[0.5cm]
        
        98057\\[2cm]
        
        \textbf{Mechanics in Selected Rhythm Games and Their Impact \\ on Immersion and Flow}\\[4cm]
        
        \normalsize
        \raggedleft
        Bachelor's thesis written under the supervision of\\
        Doctor Miłosz Markocki\\[4cm]
        \centering
        BYDGOSZCZ 2025\\[1cm]
    \end{center}
    \end{titlepage}
}
\makeatother

\begin{document}
\pagenumbering{gobble} % No page numbers initially
\maketitle

% Strona z oświadczeniem
\thispagestyle{empty}
\clearpage
% Header fields
\noindent\makebox[0.4\textwidth]{\dotfill}\\
nazwisko i imię\\[0.2cm]
\noindent\makebox[0.4\textwidth]{\dotfill}\\
nr albumu\\[0.2cm]
\noindent\makebox[0.4\textwidth]{\dotfill}\\
kierunek studiów\\[0.2cm]
\noindent\makebox[0.4\textwidth]{\dotfill}\\
typ studiów i forma kształcenia\\

% Title
\begin{center}
    \textbf{OŚWIADCZENIE}\\[0.3cm]
    \textbf{autora pracy dyplomowej}
\end{center}

\noindent
Świadomy(a) odpowiedzialności prawnej oświadczam, że praca dyplomowa\\[0.3cm]
\mbox{}\dotfill\\
\mbox{}\dotfill\\
\mbox{}\dotfill\\[0.2cm]
\noindent
została wykonana samodzielnie i nie zawiera treści uzyskanych w sposób niezgodny z obowiązującymi przepisami.

\vspace{0.5cm}
\noindent
Oświadczam również, że:

\noindent
1.~przedstawiona praca nie była wcześniej przedmiotem procedur związanych z uzyskaniem tytułu zawodowego w uczelni;

\noindent
2.~drukowana wersja pracy dyplomowej jest identyczna z wprowadzoną do systemu APD wersją elektroniczną. \\

\noindent
\phantom{dummy text}\hspace{7cm}\dotfill

\noindent
Bydgoszcz, dn.\ \makebox[0.3\textwidth]{\dotfill} \\
\noindent

\noindent Wyrażam zgodę / nie wyrażam zgody na udostępnienie przez Uniwersytet pracy dyplomowej dla potrzeb działalności badawczej i dydaktycznej. \\

\noindent
\phantom{dummy text}\hspace{7cm}\dotfill

\noindent
Bydgoszcz, dn.\ \makebox[0.3\textwidth]{\dotfill}

% Streszczenie
\chapter*{Streszczenie}
\addcontentsline{toc}{chapter}{Streszczenie}
\section*{Temat pracy licencjackiej}
\noindent
\textbf{Imię i nazwisko autora pracy:} \dotfill

\noindent
\textbf{Nr Albumu:} \dotfill

\noindent
\textbf{Imię i nazwisko promotora pracy:} \dotfill

\section*{Słowa kluczowe}
słowo1, słowo2, słowo3, słowo4, słowo5

\section*{Treść streszczenia (abstrakt)}
Tu wpisz treść streszczenia pracy. Abstrakt powinien zawierać:
\begin{itemize}
    \item główny cel pracy
    \item zastosowane metody badawcze
    \item najważniejsze wyniki
    \item główne wnioski
\end{itemize}

Abstrakt powinien zmieścić się na jednej stronie.

% Start numbering from chapter 1 with actual PDF page number
\cleardoublepage
\pagenumbering{arabic}
\setcounter{page}{5}  % Adjust this number to match your PDF

% Spis treści
\tableofcontents

\let\originalchapter\chapter

% Create a custom command for the first chapter
\newcommand{\firstchapter}[1]{%
  \titleformat{\chapter}
    {\normalfont\huge\bfseries}
    {}
    {0pt}
    {\huge}
  \chapter{#1}
  % Restore original chapter format with title on new line
  \titleformat{\chapter}[display]  % Using display style for separate lines
    {\normalfont\huge\bfseries}
    {\chaptertitlename\ \thechapter}
    {20pt}
    {\huge}
}

% Rozdziały pracy
\firstchapter{Introduction}
\label{chap:wstep}
Lorem ipsum dolor sit amet, consectetur adipiscing elit. Nullam nec purus nec nunc tincidunt ultricies. Nullam nec purus nec nunc tincidunt ultricies. Nullam nec purus nec nunc tincidunt ultricies. Nullam nec purus nec nunctincidunt ultricies. Nullam nec purus nec nunc tincidunt ultricies. Nullam nec purus nec nunctincidunt ultricies. Nullam nec purus nec nunc tincidunt ultricies. Nullam nec purus nec nunctincidunt ultricies. Nullam nec purus nec nunc tincidunt ultricies. Nullam nec purus nec nunctincidunt ultricies. Nullam nec purus nec nunc tincidunt ultricies. Nullam nec purus nec nunctincidunt ultricies. Nullam nec purus nec nunc
\section{Objectives}
Nullam nec purus nec nunc tincidunt ultricies. Nullam nec purus nec nunctincidunt ultricies. Nullam nec purus nec nunc tincidunt ultricies. Nullam nec purus nec nunctincidunt ultricies. Nullam nec purus nec nunc
\section{Methodology}
Nullam nec purus nec nunc tincidunt ultricies. Nullam nec purus nec nunctincidunt ultricies. Nullam nec purus nec nunc tincidunt ultricies. Nullam nec purus nec nunctincidunt ultricies. Nullam nec purus nec nunc

\chapter{Beginning of the genre}
\label{chap:rozdzial1}
Before one needs to understand what exactly is behind the term "rhythm game" one needs to understand the history and heritage of the genre. The first game to introduce a gameplay that somehow resembles part of mechanics of contemporary rhythm games is Simon created in 1978 by Ralph Baer and Howard Morrison. In that computer handheld game, the player needs to repeat the sentence in which the buttons light up. The sequence becomes progressively longer, up to the point where a player is unable to repeat it in the right order. There was no background music to it to speak of, so it cannot be considered a fully-fledged rhythm game, but the action of repeating sequences and patterns is a fundamental part of present-day rhythm games genre.

\begin{figure}[h]
    \centering\includegraphics[scale=0.15]{obrazki/simon.jpg}
    \caption{Electronic game Simon - It became a massive worldwide success, becoming a pop culture symbol. The game spawned many different releases and imitators with similar or same basic gameplay. \cite{simongame}}
    \label{fig:simon_game}
\end{figure}

The first rhythm game that can be recognized as such is PaRappa the Rapper (1996) published by Sony Computer Entertainment for PlayStation platform, building its core gameplay around music. Because of its unique art style, good narrative and catchy soundtrack, the game was well received among the players and critics, listed as one of the best video games ever made several times. \cite{acclaimed_videogames_parappa} This success contributed significantly to the rise in popularity of the genre. In the PaRappa the Rapper gameplay, the player must press correct buttons in response to the rhythm of currently playing track and symbols that appear on the top of the screen. The correct sequence is first performed by a teacher, after which PaRappa needs to respond accordingly. Pressing the correct buttons in accordance to the rhythm results in PaRappa rapping. One can observe the resemblance to aforementioned Simon electronic game -– PaRappa built upon this mechanic, adding additional visual and  auditory feedback, background music and grading system, setting the stage for further developments of this genre. Especially important was the addition of background music synced with other gameplay elements, making it easier to time the hits correctly. The player's final accuracy in graded from Awful to Cool –- this is currently a standard, expected element of every representant of the genre. The score is affected not only by omitted hits, but also less-than-ideal hits - the more accurate the hit, the better the rating.

\begin{figure}[h]
    \centering\includegraphics[scale=0.25]{obrazki/parappatherapper.jpg}
    \caption{A frame from the PaRappa the Rapper Remastered from 2017 showing the input guide at the top of the screen, grading and scoring system. Remastered was used here as an example, but the original had the exact same mechanics back in 1996. \cite{parappatherapper}}
    \label{fig:parappa_the_rapper}
\end{figure}

The release of Beatmania in 1997 by Konami was another milestone in the development of rhythm games genre. To enhance the player's experience with a more immersive input device, the game was introduced to Japanese arcades instead of releasing it on home platforms. Beatmania's arcade cabinet include a special input device which resemble a DJ console -– it consists of 5 buttons arranged in a piano-like pattern and round pad that mimic a vinyl record. 

\begin{figure}[h]
    \centering\includegraphics[scale=0.25]{obrazki/beatmaniacontrols.jpg}
    \caption{A controller of 1st Beatmania arcade release, showing the buttons layout and the turntable. \cite{beatmaniacontrols}}
    \label{fig:beatmania_controls}
\end{figure}

The gameplay is enclosed on a stage that consists of a vertically-scrolling panel with hit-notes on the sides, music video and audience bar in the middle. The player is supposed to press the buttons and turn the turntable accordingly with the notes that are falling from the top to the bottom of the screen, indicating the time to react when they fall down on the judgement line above illustration of the controller. This is currently known as vertical scrolling rhythm game -– which PaRappa the Rapper could not be considered as such because it showed the input sequences in batches. The game turned out to be a big hit, resulting in Konami putting more effort and resources into Konami's Games and Music Division. Due to the success of Beatmania, this department changed its name to Bemani paying respects to its prior game title. \cite{musicbasedgames} After exploring with more concepts of rhythm games, they came up with another big hit, which was Dance Dance Revolution (1998) -– pioneering title in the rhythm game genre. It adapted the basic concept of vertical scrolling rhythm game to a new form of input. The game is controlled through a dance platform, with the player standing in the middle and pressing the buttons with their feet.

The core gameplay is very akin to beatmania, sans the turntable. The VSRG formula has been adapted to a new type of play with player in standing position, dancing to the upcoming rhythm notes. This small change resulted in an experience unlike anything before, as following the patterns became even more natural and dancing became a natural result of playing the game properly.  The game became an even bigger success than beatmania, achieving worldwide popularity, being an introduction to the rhythm game genre for many players around the world. A major factor in this was definitely a natural connection between its input method and gameplay revolving around dancing.

\begin{figure}[h]
    \centering\includegraphics[scale=0.25]{obrazki/ddrplatform.png}
    \caption{A Dance Dance Revolution dance platform. \cite{ddrplatform}}
    \label{fig:ddr_platform}
\end{figure}

\begin{figure}[h]
    \centering\includegraphics[scale=0.271]{obrazki/ddrgameplay.jpg}
    \caption{Dance Dance Revolution gameplay, showing previously described gameplay elements such as health bar and hit notes with matching alignment bar - taking a form of note outline. Music video is shown playing in the background. \cite{ddrgameplay}}
    \label{fig:ddr_gameplay}
\end{figure}
\pagebreak
Taking these games as examples, the core gameplay of all rhythm games is based on performing an action accordingly to the music's rhythm and displayed pattern. It can be defined that a rhythm game is a medium that puts strong emphasis on a player's rhythm sense, coordination and reflexes. Another staple of the medium is often a presence of a custom controller, as demonstrated by beatmania and Dance Dance Revolution. This stays true to this day -– BEMANI is still publishing new installments of those franchises as of 2024, as well as other rhythm games that incorporate all aforementioned mechanics. An example of other immensely popular series featuring such would be Activision's Guitar Hero series that successfully ported the arcade experience onto the living room, bundling the game with the controller starting on the 6th generation of video game consoles.

\begin{figure}[h]
    \centering\includegraphics[scale=0.6]{obrazki/gh2bundle.jpg}
    \caption{Guitar Hero Metallica PS2 bundle, showing the game disc and plastic guitar controller. \cite{gh2bundle}}
    \label{fig:gh2_bundle}
\end{figure}

\section{Further developments}
Consecutively, rhythm games were released to both arcades and other platforms, such as home and handheld consoles or PCs. In order to relocate the experience of arcade booths into home, input devices of arcade games were adapted into home versions of dedicated controllers. For example, an open-source project StepMania, which is both a PC game and game engine at the same time, was developed with the purpose of replicating the Dance Dance Revolution experience at home. It can be played with keyboard or dance pads matching the pattern of DDR, or alternatively following Pump It Up! (Andamiro) controls -– a dance game with 4 buttons placed diagonally and one extra button in the middle.

\begin{figure}[h]
    \centering\includegraphics[scale=0.7]{obrazki/ddrsoftpad.jpg}
    \caption{A soft dance-pad which can be used to play both DDR and Pump it Up! at home, It can be plugged into a PC or a console. \cite{ddrsoftpad}}
    \label{fig:ddr_softpad}
\end{figure}

In the present day, there are many popular PC games that do not require any dedicated controllers or input devices, making it easier to entry the genre with basic gaming setup. Games such as StepMania, osu! or DJMAX, have online scoreboards and online multiplayer modes, that bring players together and create lively communities, making the competitive nature as a natural part of their gameplay and purpose for players.

\begin{figure}[h]
    \centering\includegraphics[scale=0.4]{obrazki/sm5multi.jpg}
    \caption{StepMania screenshot, showing Multiplayer Mode where 2 players are competing with each other for better score. \cite{sm5multi}}
    \label{fig:sm5_multi}
\end{figure}
\pagebreak
\begin{figure}[h]
    \centering\includegraphics[scale=0.6]{obrazki/osuleaderboards.png}
    \caption{osu! online leaderboards as of Jan 2025, showing top players performance from around the world. \cite{osuleaderboards}}
    \label{fig:osu_leaderboards}
\end{figure}
 
On the other hand, there are many arcade games that have unique controllers with various forms. 
One of such arcade game that is worth mentioning is WACCA, developed published by Marvelous which was created in collaboration with HARDCORE TANO*C, released in 2020. The game's user interface is enclosed in a circle, surrounded by a circular, segmented ring panel on the screen edges. The notes are appearing on the center screen, approaching the player through the ring panels. Because it's fun form somehow resembling a washing machine drum, this game stands out from other rhythm game arcades, as it's financially unviable to port to a home console or a PC due to this control and gameplay.
\pagebreak

\begin{figure}[h]
    \centering\includegraphics[scale=0.1]{obrazki/waccaarcade.jpg}
    \caption{Figure 10 WACCA's arcade booth, showing it's unique form and controls. \cite{waccaarcade}}
    \label{fig:wacca_arcade}
\end{figure}

The genre of rhythm games also has an immense value as educational tool, as showed through the release of Rocksmith by Ubisoft. Instead of using a fake plastic controller, the game allows the player to plug in any real electronic guitar, making it the center point of the game. The game's interface is similar to Guitar Hero's stage, but approaching notes are more specific in order to instruct player what grips and strings are supposed to be used. Due to using a real guitar instead of plastic device, the game was received as great development milestone in the genre, highlighted by the educational value provided by playing on an actual instrument, as creating a digital rhythm game with real instruments as an input device was a notable achievement for the genre. The game is developed to this day in form of Rocksmith+ -- a free-to-play title available on PC, monetized through in-app purchases of songs and lessons.

As Fayali Kares stated in his thesis "Playing Music: Design, Theory, and Practice of Music-based Games" \cite{fayeskayali}: 
\begin{quote}
    \singlespacing
    Overall, rhythm games have changed very little over time. The only changes Harmonix made to the much older Konami games concerned perspective and direct control of the underlying soundtrack. Distinction between rhythm games revolves mostly around the different input devices. From floor mats in DDR, to turntables or guitars, a variety of interfaces have found their way into the homes of rhythm game fans.
    \singlespacing
\end{quote}

Noteworthy, despite the fact that all rhythm games have similar core gameplay, with the development of technology this genre has been able to expand the player experience through the use of new input devices and unique approach to the gameplay. With the rise of such technologies as virtual reality or touch screens, rhythm game developers have utilized new solutions in order to create new experiences. Because of this, the creativity of game designers is no longer limited as it was in the past.

\chapter{Immersion in Rhythm Games}
\label{chap:rozdzial2}
The crucial part of the player engagement is the game's design and its mechanics, which need to include elements that will motivate the player to play. Such elements are universal for any other game genre, as the play needs to be meaningful. Importantly, "The meaning of a game is facilitated by design: when players can choose among playing to win, playing to keep the game interesting, or playing to manage the social situation, a game quickly become socially meaningful" (Juul 2008: 22) \cite{casualrevolution}. Because rhythm games are a specific genre, they have their own unique elements that can be used to engage the player. Something that cannot be overlooked is the adaptation of rhythm games to the current playerbase and gaming market. Considering the meanings described by Juul, the player's motivation may be as such:
\begin{itemize}
\item Playing to win - Players are driven by the need to compete and improve their ranking, often facilitated by online leaderboards, achievements, and tournaments.
\item Playing to keep the game interesting - Players are engaged by the opportunity to explore the various gamemodes, time-limited events, DLCs, or the option to create own mods and content for the community.
\item Playing to manage the social situation - the player is motivated to play by the desire to socialize and connect with other players. This can be achieved through multiplayer modes, playing in arcade parks and community events.
\end{itemize}
This classification simplifies the various motivations that drive players, and it is important to note that these motivations often overlap and can influence one another. For example, a player may be motivated to play to win, but also enjoy the social aspect of competing with friends - in such case the competetive aspect of the game may be countered by the motivation to spend quality time with friends. Additionally, the motivations can change over time, as players may become more interested in one aspect of the game as they progress or as the game evolves. The key is to create a game that offers a variety of experiences and allows players to engage with it in different ways, catering to their individual motivations and preferences.

\section{Playing to win - the competitive aspect of rhythm games}
A simple scoring system naturally enables the desire to improve and compete with other players. The competitive aspect of rhythm games is often facilitated by online leaderboards - showing best scores on the particular song and difficulty and the general ones that summarize all scores achieved. In many rhythm games, the leaderboard, which ranks players based on their highest all-time scores, lacks the element of real-time competition. Because of this, the players may be motivated to engage in tournaments that will allow them to compete with other players in real-time. Such tournaments may be organized by the developers or held unofficially by the community. Usually, the tournaments organized by the developers focus on the best players, while the community tournaments are more casual and focus on the fun of playing together, often including players from lower-rank brackets. Such form of play measures different skills than simply playing the game by itself, as it deprives the player from the ability to retry the song after missing a note or failing, which is a common practice in the single-player mode. In this setting, players are required to complete the song in a single attempt, which can be particularly challenging for those accustomed to repeating tracks multiple times to achieve a perfect score. Therefore, even if the tournament is held online, it takes a different form than competing for the best score on the leaderboard.

\section{Playing to keep the game interesting - outside the core gameplay}
As current gaming market makes the biggest profit from live service games, the genre of rhythm games adaptates this idea to its own needs. The live service model allows developers to keep the game fresh and engaging for players by regularly releasing new content, such as songs, gamemodes, and time-limited events. This solution assures to keep the player interest and encourage them to continue playing over time. In this case, rhythm game developers focus on releasing DLCs with new songs and extra difficulties, often including recently released songs from popular artists or franchises, such as Vocaloid. Additionally, many rhythm games also include time-limited events, which can create a sense of urgency and encourage players to log in and play regularly. For example, in arcade rhythm games such as \textit{maimai}, the player receives in-game currency for daily log-in from IC Card and completing current time-limited challenges. Additionally, the owner of the cab needs to update it regularly, as the developer continues to support the game and release the new updates with additional content. This solution is different than old-school arcades that were released once, completed on the day of the release as the new version came only alongside a completly new cab release. This day, as the cabs are connected to the internet, the new content is distributed online and arcade parks can easily update the game. Nowadays, the new cab is released only when the hardware of the old one is insufficient - for example, \textit{maimai} continues to receive new updates only on the newer version of its arcade machine. The first generation of the cabinet received 12 updates between the year 2012 and 2019 before the introduction of the new mechanic: tap-notes, which needed the upgraded version of the touchscreen. The second generation of the arcade machine with upgraded touchscreen was released as \textit{maimai DX} in July 2019. Currently, the second generation of \textit{maimai DX} machines receives new updates once per few months. Between the releases, the playerbase receives time-limited events with missions, achievements, profile titles and cosmetics.

\section{Playing to manage the social situation - the culture of rhythm games and its community}
The social aspect of rhythm games is often overlooked, but it is an important part of the player experience. Many players enjoy playing rhythm games with friends or in a social setting, such as an arcade park or online multiplayer mode. As Alexander Chan highlights: "Throughout the history of video and computer games, game communities have often been as interesting to examine as the games themselves. They form an integral part of the entire game environment and without examining the game community, any exploration of a game would be incomplete" (Chan 2004: 6) \cite{arcadeculture} - rhythm games are no exception. Early rhythm game communities formed in arcade parks, where players gathered locally in order to play on cabs. With the rise of PC rhythm games and the growth of online forums, these communities gradually transitioned to the internet. Today, the social aspect is further enhanced by social media, community platforms, and chatrooms, allowing players to share experiences and build connections, that further enhance the sense of belonging to the community. Since the early days of the genre, social interaction has played a key role, contributing to the development of a distinct culture around rhythm games. In the early stages, this culture was heavily influenced by japanese pop culture and platforms like NicoNicoDouga, which hosted a large volume of fan-made content that developers and players integrated into rhythm games. AIn the early stages, this culture was heavily influenced by pop culture and platforms like NicoNicoDouga, where users shared a large volume of fan-made content that developers and players later integrated into rhythm games. One example is the MAD video format—fan-made remixes that combine cutout fragments from existing videos of other media, such as anime, films, or advertisements, often featuring recognizable characters. One of the primarly example is a MAD video known as "Ronald McDonald Insanity", which is a remix of a song \textit{U.N. Owen Was Her?} from \textit{Touhou Project} - featuring rapid cuts of Ronald McDonald from the McDonald's commercials. The video was released in 2007 on NicoNicoDouga and quickly gained popularity, making its way into various rhythm games, including \textit{Stepmania} and \textit{osu!}. Inrestingly, \textit{Touhou Project} music and its fan-made content is commonly featured in rhythm games. The popularity of \textit{Touhou Project} music in rhythm games can be attributed to the fact that the franchise has a large and diverse library of songs, which are often remixed and rearranged by its talented and creative fans. The community-driven nature of rhythm games and their intersection with other fandoms have also contributed to the popularity of other forms of media, such as Vocaloid - voice-synthesizing software used to create songs, often featuring virtual singers, such as Hatsune Miku, the most popular one. Rhythm games, especially \textit{Hatsune Miku: Project Diva} series, played a significant role in popularizing Vocaloid music globally introducing it to the larger audience. Overall, the social aspect of rhythm games is a crucial part of the player experience, as it allows players to not only connect with others, but also creates an opportunity to interact with other media and find new interests.

\chapter{Podsumowanie}
\label{chap:podsumowanie}
Fundamentally, the concepts that started the rhythm games genre can be tracked back to the early electronic game \textit{Simon} and the PlayStation game from 1996 -- \textit{PaRappa The Rapper}, which introduced a gameplay that revolves around pressing buttons to the rhythm of the song that is currently played in the background. This concept was further followed by musical games -- such as \textit{beatmania} and \textit{DanceDanceRevolution}, which introduced dedicated controllers for the player's input. These games established the main centerpiece of the genre -- the vertical-scrolling formula for showing notes that correspond to the buttons of dedicated input devices. This aspect was further used for another rhythm games that played a big part in the process of popularization of rhythm games, such as \textit{Guitar Hero}. The case study of \textit{Guitar Hero} demonstrates the importance of the relationship between music and the audience, as it broadened the appeal to the western market.

Wile the core gameplay of rhythm games still revolved around the player's reflexes and input synchronized with the music's rhythm, the progress of technology and advancements in the gaming market allowed the game developers to expand the player's interaction with music, not only in arcades, but also at home. Essentially, the development of feedback systems played the crucial part in reinforcing the player's immersion and reinforcing the engagement. The case study of \textit{Beat Saber} and \textit{SOUND VOLTEX} shows that the effective combination of visual, auditory and tactile feedback, significantly enhances core gameplay elements, which naturally makes it easier to immerse the player into the game and evoke the state of flow.Analyzing modern rhythm games illustrates that the player's interaction with music is enhanced by the combination of unique controllers, audio-visual effects and feedback systems, which instrumentally create a powerful sense of performing the music. Such immersive aspects directly support the necessary conditions for evoking the flow state, providing immediate feedback that satisfies and engages the player. The evocation of flow state is further enhanced by allowing the player to individually choose the desired difficulty, making it easier to balance between the challenge and boredom. Additionally, it makes the player always aware of their current skill. As rhythm games enable the capability to balance independently between the challenge and boredom, the genre has an infinite resource of replayability. Due to the fact that music is pleasant to listen to on its own, the player can always choose the desired level of challenge and focus on the interaction with music. Noteworthy, the level design and designed patterns highlight the progression of currently played song - the idea of enrichment of music makes it even more enjoyable for the player, therefore making the gameplay even more immersive. Knowing that fact, game developers can include music genres that will resonate with their target audience the most. As a result, the goal of playing the game can be simply brought to enjoying the process of interaction with music, getting better scores and skill progression. What's more, the community of rhythm games is encouraged to create own content, as some rhythm games provides a platform to upload own difficulties, songs and UI cosmetics -- which is illustrated y the example of \textit{osu!}. In order to keep the game interesting, some games include progression systems and hidden content that has to be unlocked throughout the play -- this practice is used in \textit{Groove Coaster: Wai Wai Party!!!!} for Nintendo Switch. It might seem that rhythm games does not include story modes or quests, but the \textit{Guitar Hero} shows otherwise -- as it includes a developed story mode, requiring the player to progress as a rock band member and play at revenues. This aspect of clear goals coincidences with components required to achieve the flow state. According to Mihaly Csikszentmihaly, the components of flow state are always the same, no matter the task. Besides the balance between the challenge and boredom, the direct feedback and sense of clear goals, the task must absorb into the activity and allow them to fully concentrate. As a result, the element of time transformation and the loss of self-consciousness should take place. In case of rhythm games, these conditions take place during the immersion and concentration on the play. When examining the arcade rhythm games, it is visible that the environment of arcade parks is arranged in such a way, that the player is surrounded by conducive factors. Arcades also ring out the social part of the play -- case study of \textit{maimai} game series shows that the arcade rhythm game can encourage its players to play together, as it features both VS and Sync Mode. Naturally, the competitive aspect also evolved out of arcades, bringing in community events such as \textit{osu! World Cup}, along with fan-made content that nourished the culture around the genre. Together, the rhythm games are a product of well designed mechanics, enjoyable music, psychological factors and surrounding culture that will stay alive due to passionate communities and competitive structures.

% Bibliografia

\printbibliography[heading=bibintoc,title=Bibliography]

% Spis rysunków
\listoffigures
\addcontentsline{toc}{chapter}{List of Figures}

% Spis tabel
%\listoftables
%\addcontentsline{toc}{chapter}{Spis tabel}

% Spis listingów
%\lstlistoflistings
%\addcontentsline{toc}{chapter}{Spis listingów}

\end{document}
