Video games, as a genre, incorporate a vast array of gameplay experiences, each offering unique interactive gameplay. Among these, rhythm games stand out for their direct engagement with music, requiring players to perform actions in synchronization with auditory and visual cues. From the simple pattern repetition of early electronic toys to the complex, multi-sensory experiences offered by modern arcade cabinets and virtual reality setups, rhythm games have defined a distinct niche within the interactive entertainment landscape. Their appeal lies not just in the challenge they present, but in their ability to evoke a powerful sense of connection to the music, often leading to states of deep immersion and enjoyment.
This thesis examines the world of rhythm games, seeking to understand the elements that define this genre and contribute to immersion and flow state in them. It explores the historical trajectory of rhythm games, tracing their evolution from inceptive concepts to broad gameplay systems and varied design approaches. Central to this exploration is an analysis of the core mechanics and, crucially, the feedback systems -- visual, auditory, and tactile -- that translate player input into meaningful interaction and create the feeling of musical performance.
Furthermore, this work examines the core principles behind the rhythm game experience, focusing particularly on the concept of ``flow'' -- the state of complete absorption in an activity. It investigates how game design elements and gameplay mechanics are strategically employed in rhythm games to facilitate and maintain the state of flow.
Finally, the thesis acknowledges that rhythm games are influenced by external factors. It explores the vital role of player engagement beyond the core gameplay loop, considering competitive motivations, the impact of post-launch content support, and the significant social and cultural dimensions that shape rhythm game communities. Such aspects as dedicated arcade scenes, thriving online platforms and intersections with fan cultures, play a crucial role in the longevity and relevance of rhythm games genre. 
To achieve this analysis, the subsequent chapters are structured as follows: Chapter 1: ``Rhythm Games'' provides a historical overview, outlining the origins and key milestones in the development of the rhythm game genre. It also describes the mechanisms of immersion, breaking down how feedback systems create engaging player experiences. Afterwards, it analyzes the theory of flow state and its relevance within rhythm game design. Chapter 2: ``Player’s engagement: social and cultural aspects of rhythm games'' examines player engagement through the lens of competition, game updates, and the crucial social and cultural aspects of the rhythm game community.
In order to analyze the fundamental aspects of rhythm games, the thesis draws on a range of sources, including academic literature and articles. To illustrate key concepts and provide practical examples of how these elements manifest in gameplay, the thesis also incorporates case studies of specific rhythm games.
The purpose of this research is to examine how specific gameplay mechanics in rhythm games contribute to player immersion and evoke flow state, offering insights into how these experiences are constructed and how do they influence the player’s engagement.