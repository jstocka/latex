Various representatives of video games genre offer a vast array of mechanics, resulting in unique experiences. Among these, rhythm games tend to stand out thanks to their direct interaction with music, naturally putting players in a position where they need to organically interact with the medium in tandem with auditory and visual cues. Beginning with simple toys that put pattern recognition and memory skills to the test, rising from the humble arcade beginnings, modern examples of the genre form a distinct niche in the gaming landscape. Their appeal lies not only in the presented challenge, but also in their competitive nature (also inherited from arcade-born tournaments), ability to evoke a strong connection to the music, resulting in deep player immersion.

This thesis looks at the landscape of the rhythm games, considers the elements that together create this genre. It explores the history starting from the most basics concepts, following with their evolution, leading to modern gameplay loops by exploring and dissecting the most important examples. The analysis focuses on core gameplay mechanics, visual, auditory and tactile feedback systems present in controllers and games themselves and how their all work together in order to create an experience unlike any other, resembling a musical performance.

A strong focus was put into examining the core building blocks of this experience, particularly on the concept of ``flow'', where the player is completely immersed in the game, with the controls feeling like a natural extension of their body and mind. Examples of design elements and mechanics are shown as main contributors in achieving this goal. 

Factors outside the core gameplay, such as the competitive aspect, the social communities that formed around those games, live service model mechanics and post-launch support are also considered and analyzed. All of them significantly help to keep the genre alive and interesting, keeping player engagement high while keeping the core gameplay intact.

The subsequent chapters are structured as follows: Chapter 1: ``Rhythm Games'' provides a historical summary, bringing up case studies of various examples to outline key milestones during the development of the genre. It analyses how all aforementioned mechanics contribute to create an engaging gameplay loop. Flow state is also brought up studied, finally shown as inseparable outcome of everything coming together, defining the genre. Chapter 2: ``Player’s engagement: social and cultural aspects of rhythm games'' takes a look at all the other aspects contributing to the popularity of those games, taking a look at social and cultural aspects. It analyses player motivations, expands on the competitive side and its arcade roots, and looking what is possible with modern tools and technology. Live service aspects are also studied, along with post-launch developments.

To achieve the goals the thesis draws on diverse sources, from academic literature and articles to community-born recordings of events and developer-published materials. Gameplay fragments and pictures of the controllers are used as they are necessary to fully present a point in each case study. 

Examination of all of the above allows to reach a conclusion in how gameplay mechanics evoke the flow state, reach upmost level of player immersion, and contribute to strong player engagement, both during the gameplay and outside of it, as a part of a community.