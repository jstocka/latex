Fundamentally, the concepts that started the rhythm games genre can be tracked back to the early electronic game \textit{Simon} and the PlayStation game from 1996 -- \textit{PaRappa The Rapper}, which introduced a gameplay that revolves around pressing buttons to the rhythm of the song that is currently played in the background. This concept was further followed by musical games -- such as \textit{beatmania} and \textit{DanceDanceRevolution}, which introduced dedicated controllers for the player's input. These games established the main centerpiece of the genre -- the vertical-scrolling formula for showing notes that correspond to the buttons of dedicated input devices. This aspect was further used for another rhythm games that played a big part in the process of popularization of rhythm games, such as \textit{Guitar Hero}. The case study of \textit{Guitar Hero} demonstrates the importance of the relationship between music and the audience, as it broadened the appeal to the western market.

Wile the core gameplay of rhythm games still revolved around the player's reflexes and input synchronized with the music's rhythm, the progress of technology and advancements in the gaming market allowed the game developers to expand the player's interaction with music, not only in arcades, but also at home. Essentially, the development of feedback systems played the crucial part in reinforcing the player's immersion and reinforcing the engagement. The case study of \textit{Beat Saber} and \textit{SOUND VOLTEX} shows that the effective combination of visual, auditory and tactile feedback, significantly enhances core gameplay elements, which naturally makes it easier to immerse the player into the game and evoke the state of flow.Analyzing modern rhythm games illustrates that the player's interaction with music is enhanced by the combination of unique controllers, audio-visual effects and feedback systems, which instrumentally create a powerful sense of performing the music. Such immersive aspects directly support the necessary conditions for evoking the flow state, providing immediate feedback that satisfies and engages the player. The evocation of flow state is further enhanced by allowing the player to individually choose the desired difficulty, making it easier to balance between the challenge and boredom. Additionally, it makes the player always aware of their current skill. As rhythm games enable the capability to balance independently between the challenge and boredom, the genre has an infinite resource of replayability. Due to the fact that music is pleasant to listen to on its own, the player can always choose the desired level of challenge and focus on the interaction with music. Noteworthy, the level design and designed patterns highlight the progression of currently played song - the idea of enrichment of music makes it even more enjoyable for the player, therefore making the gameplay even more immersive. Knowing that fact, game developers can include music genres that will resonate with their target audience the most. As a result, the goal of playing the game can be simply brought to enjoying the process of interaction with music, getting better scores and skill progression. What's more, the community of rhythm games is encouraged to create own content, as some rhythm games provides a platform to upload own difficulties, songs and UI cosmetics -- which is illustrated y the example of \textit{osu!}. In order to keep the game interesting, some games include progression systems and hidden content that has to be unlocked throughout the play -- this practice is used in \textit{Groove Coaster: Wai Wai Party!!!!} for Nintendo Switch. It might seem that rhythm games does not include story modes or quests, but the \textit{Guitar Hero} shows otherwise -- as it includes a developed story mode, requiring the player to progress as a rock band member and play at revenues. This aspect of clear goals coincidences with components required to achieve the flow state. According to Mihaly Csikszentmihaly, the components of flow state are always the same, no matter the task. Besides the balance between the challenge and boredom, the direct feedback and sense of clear goals, the task must absorb into the activity and allow them to fully concentrate. As a result, the element of time transformation and the loss of self-consciousness should take place. In case of rhythm games, these conditions take place during the immersion and concentration on the play. When examining the arcade rhythm games, it is visible that the environment of arcade parks is arranged in such a way, that the player is surrounded by conducive factors. Arcades also ring out the social part of the play -- case study of \textit{maimai} game series shows that the arcade rhythm game can encourage its players to play together, as it features both VS and Sync Mode. Naturally, the competitive aspect also evolved out of arcades, bringing in community events such as \textit{osu! World Cup}, along with fan-made content that nourished the culture around the genre. Together, the rhythm games are a product of well designed mechanics, enjoyable music, psychological factors and surrounding culture that will stay alive due to passionate communities and competitive structures.