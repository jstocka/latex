\begin{center}
   \textbf{\large Mechanics in Selected Rhythm Games and Their Impact on Immersion and Flow}
\end{center}
Name and surname of the author: Julia Stocka \\
Student ID number: 98057 \\
Thesis Supervisor: Dr. Miłosz Markocki

\section*{Keywords}
\noindent
Flow, Immersion, Rhythm Games, Music in games, Gameplay Mechanics, Player Engagement, Competitive Gaming

\section*{Abstract in English}
\noindent
The paper examines the impact of rhythm game mechanics on the sense of immersion present in rhythm games and the initiation of a flow state in the player. The research methods include an evaluation of the history of the genre, the competitive aspect present in games, selected mechanics in the representatives of the genre and the visual, auditory and tactile feedback systems present in them. By performing case studies of selected games, the key findings demonstrate the joint effect and symbiosis of the mentioned factors on the process of achieving a state of flow and player engagement. The possibilities of adjustments of the game objective, level of difficulty and variety of songs in rhythm games further enhance this state. Social and cultural factors common in the representatives of the genre are also strengthening in this aspect. The conclusion of the thesis points to the satisfying entertainment loop and rewarding feedback, intrinsically building engagement and flow, as a major factor in the appeal of the rhythm game genre.

\section*{Abstract in Polish}
Praca poddaje badaniu wpływ mechanik gier rytmicznych na występujące w nich poczucie immersji oraz wywołaniu stanu flow u gracza. Metody badawcze obejmują przegląd historii gatunku, aspektu rywalizacji obecnego w grach, wybranych mechanik w reprezentantach gatunku oraz obecnych w nich systemów informacji zwrotnej - wizualnych, dźwiękowych oraz dotykowych. Poprzez studium przypadku wybranych gier kluczowe wyniki ukazują wspólne działanie oraz symbiozę wyżej wymienionych czynników na proces osiągania stanu flow i zaangażowania gracza. Możliwości w zakresie dostosowywania celu gry, poziomu trudności oraz różnorodności utworów w grach rytmicznych dodatkowo wzmagają ten stan. Budujące w tym aspekcie są także czynniki społeczne i kulturowe, będące wspólnym punktem w przedstawicielach gatunku. Konkluzja pracy wskazuje na satysfakcjonującą pętlę rozrywki oraz nagradzające informację zwrotne, w sposób nieodłączny budujące zaangażowanie i flow, jako główny czynnik stanowiący o atrakcyjności gatunku gier rytmicznych.