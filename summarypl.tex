\begin{center}
   \textbf{\large Mechaniki w wybranych grach rytmicznych oraz ich wpływ na immersję i flow}
\end{center}
Imię i nazwisko autora pracy: Julia Stocka \\
Nr Albumu: 98057 \\
Imię i nazwisko promotora pracy: dr Miłosz Markocki

\section*{Słowa kluczowe}
\noindent
Flow, immersja, gry rytmiczne, muzyka w grach, mechaniki w grach, zaangażowanie gracza, gry kompetytywne

\section*{Abstrakt}
Praca poddaje badaniu wpływ mechanik gier rytmicznych na występujące w nich poczucie immersji oraz wywołaniu stanu flow u gracza. Metody badawcze obejmują przegląd historii gatunku, aspektu rywalizacji obecnego w grach, wybranych mechanik w reprezentantach gatunku oraz obecnych w nich systemów informacji zwrotnej - wizualnych, dźwiękowych oraz dotykowych. Poprzez studium przypadku wybranych gier kluczowe wyniki ukazują wspólne działanie oraz symbiozę wyżej wymienionych czynników na proces osiągania stanu flow i zaangażowania gracza. Możliwości w zakresie dostosowywania celu gry, poziomu trudności oraz różnorodności utworów w grach rytmicznych dodatkowo wzmagają ten stan. Budujące w tym aspekcie są także czynniki społeczne i kulturowe, będące wspólnym punktem w przedstawicielach gatunku. Konkluzja pracy wskazuje na satysfakcjonującą pętlę rozrywki oraz nagradzające informację zwrotne, w sposób nieodłączny budujące zaangażowanie i flow, jako główny czynnik stanowiący o atrakcyjności gatunku gier rytmicznych.